\documentclass{acmsiggraph}               % final

%% These two line bring in essential packages: ``mathptmx'' for Type 1
%% typefaces, and ``graphicx'' for inclusion of EPS figures.


\usepackage{graphicx}
\usepackage{url}
\usepackage{times}
\usepackage{amssymb}



%% Paper title.

\title{Using Marching Cubes to generate terrain on the GPU}

%% Author / Affiliation (single author).

%%\author{Roy G. Biv}
%%\affiliation{Allied Widgets Research\thanks{email:roy.g.biv@aol.com}}

%% Author / Affiliation (multiple authors).

\author{Otto Nordander\thanks{e-mail: avo10ono@student.lu.se} \and Gustaf Oxenstierna\thanks{his13gox@student.lu.se}
}
\affiliation{Lund University\\ Sweden}


%% Keywords that describe your work.
\keywords{}

%%%%%% START OF THE PAPER %%%%%%

\begin{document}

\ifpdf
  \DeclareGraphicsExtensions{.jpg,.pdf,.mps,.png}
\else
  \DeclareGraphicsExtensions{.eps}
\fi


\maketitle

\begin{abstract}
\end{abstract}


\section{Introduction}
In this project we implemented the Marching cubes algorithm and used it to procedurally
generate random terrain using the parallel computing power of the GPU. We chose this project
because we are interested in different approached to procedurally generate terrain and we selected
this particular project because it is possible to generate this terrain in real time and at
interactive frame rates. During implementation we discovered that our textures became warped
so we also implemented tri-planar texturing to solve this issue.
\section{Algorithms}
We used two main algorithms for this project, Marching cubes and tri-planar texturing, with
the former being used to generate terrain and the latter being used to counteract the stretching
of textures applied to the terrain.
\subsection{Marching cubes}
For the Marching cubes algorithm to work, one needs a density function that describes the surface
one is trying to generate. This density function takes any world space coordinate and maps it to
a floating point number such that:
$$ f(x,y,z) : \mathbb{R}^{3} \rightarrow \mathbb{R} $$
To turn this in to something that is actually useful to us, we define the world space coordinates
where the density function is zero to be the surface we are generating. Any points where the density 
function is negative are located above the surface with a positive density function means the point
is under the surface ~\cite{Geiss}.

To generate a surface, we must first divide the terrain into an infinite number of equally sized
blocks that are further subdivided into 32x32x32 voxels. We generate our triangles by visiting
each voxel in a block and sampling the density function at all eight corners of the voxel. The
eight resulting density values are masked together to produce an 8-bit integer, denoted the case number which is used
for a table lookup to tell us how many triangles to generate for this voxel. A second table
lookup is then done, using the number of triangles and the case number to determine where in the
voxel the triangles should be located. All triangles have their endpoints connected to the corner
edges of the voxel ~\cite{Geiss}.

To actually generate the triangles, we use the geometry shader which is the only shader that can
create geometry. The density function we use is based on random noise which is sampled using
bilinear interpolation to ensure a smoother noise texture ~\cite{lodev}. The baseline for our
terrain is simply the negative y axis world space coordinate, we then add noise sampled using
different frequencies and amplitude to the density to obtain our final value.

\subsection{Tri-planar texturing}
One issue we faced when generating terrain was correctly texturing this terrain as the height
differences lead to unfavourable stretching of the textures. Tri-planar texturing is a method
for solving these issues by rendering the texture three times in three different directions:
once along the X-axis, once along the Y-axis and finally once along the Z-axis. The results
are subsequently blended together based on how much the fragment is facing a certain axis. For
instance, if a fragment is facing 25\% towards the X-axis and 75\% towards the Z-axis, 25\% of
the X-axis rendering and 75\% of the Z-axis rendering will be used ~\cite{Owens2014}.
Another advantage to using this method is that we do not have to derive texture coordinates for
the terrain since the world space coordinates are used for the texture lookup.

\subsection{Phong Shading}
For lighting we implemented a simple Phong model, similar to the labs in the EDA221 course. A
single light source is used for light calculations.

\section{Results}

\section{Discussion}

\section{Conclusion}
%Some groups may not have enough important material to write a ``Discussion''-section,
%and in such cases, that section can be omitted. Also, look at scientific papers,
%e.g.,~\cite{Hakura97,Igehy98,RaganKelley2011,Doggett2012,Ganestam2015}, and try to follow their general style.
%When in doubt, come and ask us.
%If you need references not found in the file \texttt{project.bib}, simply add your reference to
%that file. The report should be 2--4 pages long.

%You can include screenshots as PNGs or illustrations in the PDF format.
%An example is shown in Figure~\ref{fig_lugg}. See the source file \texttt{project.tex}
%for how this is done in \LaTeX.
%\begin{figure}[tb]
%    \centering
%    \includegraphics[width=0.7\columnwidth]{lugg.png}
%    \caption{This is the logotype for LUGG: Lund University Graphics Group.}
%    \label{fig_lugg}
%\end{figure}

\bibliographystyle{acmsiggraph}
%\nocite{*}
\bibliography{project}
\end{document}
